% set the document type and base font size
\documentclass[17pt]{extarticle}

% Geometry package supports custom border and format definitions
\usepackage[a4paper, top=2cm, bottom=2cm, left=2cm, right=2cm]{geometry}
% extsizes package allows to set the base font size to 8-20 pixels (https://ctan.org/pkg/extsizes)
\usepackage{extsizes}
% remove limitation to ascii by introducing UTF8 support 
\usepackage[utf8]{inputenc}
% enables support for the german 'Umlaute': ä, ö, ü
\usepackage[T1]{fontenc}
% mathematical expressions with tabstop support (https://texblog.org/2008/10/01/adding-normal-text-into-formulas/)
\usepackage{amsmath}
% text coloring, obviously
\usepackage{color}
% import and positioning of pictures and graphics
\usepackage{graphicx}

% disable indentation for new paragraphs
\setlength{\parindent}{0pt}

% document meta information
\author{Lukas Steiger}
\title{HSR Formelsammlung PhAI}
\date{20.4.17}

% start of the document content
\begin{document}
\begin{center}
	\huge{HSR Formelsammlung PhAI}
\end{center}
	
\section{Grundlagen - Bewegung und Kräfte}

	Geschwindigkeit
	\begin{align}
		v = v_{0} * a * t
	\end{align}
	
	Kraft \small{(Newtonsches Bewegungsgesetz)}
	\begin{align}
		F = m * a
	\end{align}
	
	Schwerkraft \small{(Gravitationskraft)}
	\begin{align}
		&F_{G} = m * g
		&&g = 9.81 \frac{m}{s^{2}} 
	\end{align}
	
	Zurückgelegte Strecke \small{($s_{0}$ oder $v_{0}*t$ weglassen, falls 0)}
	\begin{align}
		&s(t) = s_{0} + v_{0} * t + \frac{a}{2} * t^{2}
		&&\text{\small{oder einfach}}
		&&&s = v * t
	\end{align}

	\subsection{Würfe}

	\subsubsection{Senkrechter Wurf}
	\begin{align}
		y * v_{0} * t - \frac{g}{2} * t^{2}
	\end{align}
	
	Wurfhöhe
	\begin{align}
		h = \frac{v_{0}^{2}}{2g}
	\end{align}
	
	\subsubsection{Horizontaler Wurf}
	\begin{align}
		y = - \frac{g}{2v_{0}^{2}} * x^{2}
	\end{align}
	
	Wurfweite
	\textcolor{red}{Gibts da eine Formel?}
	
	\subsubsection{Schiefer Wurf}
	\begin{align}
		y(x) = x * \tan \varphi - \frac{g}{2 * v_{0}^{2} * \cos^{2} \varphi} * x^{2}
	\end{align}
	
	Wurfhöhe
	\begin{align}
		y_{max} = \frac{v_{0}^{2} * \sin^{2} \varphi}{2g}
	\end{align}
	
	Wurfweite
	\begin{align}
		d = \frac{v_{0}^{2} * \sin(2\varphi)}{g}
	\end{align}

\end{document}
